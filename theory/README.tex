\documentclass{article}

\usepackage{biblatex}
\addbibresource{README.bib}

\usepackage{amsmath}
\usepackage{amssymb}
\usepackage{mathpartir}

\usepackage{todonotes}
\usepackage{hyperref}

\title{A Directed, Topos-Theoretic Type Theory}
\author{finegeometer}

\begin{document}
\maketitle

% \tableofcontents

\begin{quotation}
    In anything at all, perfection is finally attained not when there is no longer anything to add, but when there is no longer anything to take away, when a body has been stripped down to its nakedness.

    --- Antoine de Saint-Exupéry; \emph{Wind, Sand and Stars}
\end{quotation}

I describe a system that displays aspects of modal and directed type theories,
at no more theoretical complexity cost than Martin L\"of Type Theory.

\tableofcontents

\section{Motivations}

\subsection{Multimode}

Standard, Martin-L\"of type theory can be interpreted in an arbitrary topos.
By working in these toposes, and adopting the axioms they support,
we get simple and beautiful ``synthetic'' accounts of various areas of mathematics,
such as differential geometry, algebraic geometry, domain theory, or higher category theory. 

In 2020, Gratzer, Kavvos, Nuyts, and Birkedal introduced \emph{Multimode Type Theory} (\cite{MTT}),
which is to be interpreted not in a single topos, but a collection of them.
It innovates over previous modal type theories in that it isn't tied to a specific interpretation.
In fact, thanks to a followup paper by Shulman (\cite{MTT_Shulman}), we know that it can be interpreted
in \emph{any} finite diagram of toposes, chosen in advance.

But why should we require the toposes be chosen in advance?
Why not let the \emph{programmer} choose, on a program-by-program basis?

\subsection{Directed}

Homotopy type theory is an extension of Martin-L\"of type theory,
in which every construction provably respects isomorphism.
That is, we can prove that every function from e.g. groups to sets
takes isomorphic groups to isomorphic sets. This is really convenient,
since otherwise we'd have to prove this manually for every function we care about;
a task that quickly becomes tedious.

But just as often, we want to know that our constructions are functorial.
That they preserve not just isomorphisms, but morphisms.
This has inspired the search for a \emph{directed} type theory;
one in which types automatically carry the structure of a category,
and functions between them are always functorial.

But there's a problem. The type \(A \to B\) isn't functorial in \(A\)!
It's \emph{contravariantly} functorial. So it seems any directed type theory
needs to handle multiple variances of functor.
This becomes more and more complicated when we start to work with higher categories,
and it is unclear how to proceed.

\subsection{Simplicity}

My final motivation is a strong desire for simplicity.

Martin-L\"of type theory is an impressively simple language.
Multimode type theory isn't \emph{too} complicated, but it's certainly less simple.
In tackling the problems described above, I am unwilling to allow
the complexity of the language to continue to increase.

As an example of this philosophy, consider Multimode Type Theory.
It has modalities, which allow the user to move to another topos.
But abstracting over a variable means you move to a slice topos.
Yet these two features are unrelated in Multimode Type Theory,
so I naturally focus on trying to merge them.

\section{Design Process}

I began with the insight was that describing a \emph{classifying} topos feels similar to describing a context in a type theory.
Describing a geometric morphism between classifying toposes feels like a substitution.

So why not merge the concepts?

Interpret the contexts of Martin-L\"of type theory as \emph{toposes}. The substitutions as geometric morphisms.
The types as the objects of the toposes, and the terms as global elements.

So I tried that. It didn't work. In the model, \((\Pi x : A. B)[\sigma]\)
and \(\Pi x : A[\sigma]. B[\sigma[\mathbf{wk}], x]\) don't have the same interpretation.

But do we need them to?

Let's simply \emph{remove} the commutation rules.
Substitutions will no longer commute through \(\Pi\)-types or lambdas.
To compensate, we generalize \(\beta\)-reduction to handle the case
where a substitution is sandwiched between the lambda and the application.

Is this \emph{too} simple? Too weak? Maybe, but maybe not.
The weakened substitution rules do not block evaluation;
they only weaken type equality.

And in removing these commutation rules,
I seem to have unintentionally solved the variance issue for directed type theory.
When we ask if \(A \to B\) is functorial in \(A\), we're no longer asking
if a map \(A \to A'\) yields a map \((A \to B) \to (A' \to B)\).
We're asking if it yields a map \((X \to B)[A/X] \to (X \to B)[A'/X]\).
Without the commutation rules, this is a different question,
and there's no reason it isn't possible.

\section{The Language}

\newcommand{\Cx}{\mathsf{Cx}}
\newcommand{\Sb}{\mathsf{Sb}}
\newcommand{\Ty}{\mathsf{Ty}}
\newcommand{\Tm}{\mathsf{Tm}}

The language is specified as follows. Only the last bullet point is different from Martin-L\"of type theory.

\begin{itemize}
    \item We have a category of contexts.
    \begin{mathpar}
        \inferrule{ }{\Cx}
        \and \inferrule{\Gamma : \Cx \\ \Delta : \Cx}{\Sb(\Gamma, \Delta)}
        \\ \inferrule{\Gamma : \Cx}{\mathsf{id} : \Sb(\Gamma,\Gamma)}
        \and \inferrule{\sigma : \Sb(\Gamma,\Delta) \\ \tau : \Sb(\Delta,\Xi)}{\tau[\sigma] : \Sb(\Gamma,\Xi)}
        \\ \mathsf{id}[\sigma] = \sigma
        \and \sigma[\mathsf{id}] = \sigma
        \and \upsilon[\tau[\sigma]] = \upsilon[\tau][\sigma]
    \end{mathpar}
    \item We have types and terms, acted on contravariantly by substitutions.
    \begin{mathpar}
        \inferrule{\Gamma : \Cx}{\Ty(\Gamma)}
        \and \inferrule{A : \Ty(\Gamma)}{\Tm(\Gamma,A)}
        \\ \inferrule{\sigma : \Sb(\Xi,\Gamma) \\ A : \Ty(\Gamma)}{A[\sigma] : \Ty(\Xi)}
        \and \inferrule{\sigma : \Sb(\Xi,\Gamma) \\ a : \Tm(\Gamma, A)}{a[\sigma] : \Tm(\Xi, A[\sigma])}
    \end{mathpar}
    \item A few specific contexts, characterized by their mapping-in behavior.
    \begin{mathpar}
        \inferrule{ }{\Gamma : \Cx}
        \and \inferrule{\Gamma,\Delta : \Cx \\ A : \Ty(\Gamma)}{(\Gamma, x : A) : \Cx}
        \\ \Sb(\Xi,\epsilon) \simeq \mathbf{1}
        \and \Sb(\Xi,(\Gamma, x : A)) \simeq (\sigma : \Sb(\Xi,\Gamma)) \times \Tm(\Xi,A[\sigma])
        \\ ()[\sigma] = ()
        \and (\tau,a)[\sigma] = \tau[\sigma],a[\sigma]
    \end{mathpar}
    \begin{itemize}
        \item In those last two equations, the computation rules for composition of substitutions,
        I conflate \(\Sb(\Xi,\epsilon)\) with \(\mathbf{1}\), and similarly for \(\Sb(\Xi,(\Gamma, x : A))\).
        I continue to do so throughout this writeup.
        \item Somewhat surprisingly, the above rules implicitly introduce the concept of variables.
        The identity substitution on any nonempty context decomposes as \(\mathsf{id} = (\mathsf{wk},\mathsf{v}_0)\),
        into a weakening substitution and the variable with DeBruijn index zero.
        If the weakened context is still nonempty, then we further decompose \(\mathsf{wk} = (\mathsf{wk}^2,\mathsf{v}_1)\).
        Et cetera.
        \todo{Where did I learn this trick? Cite it.}
    \end{itemize}
    (Those last two equations, the computation rules for composition of substitutions,
    only make sense after implicitly casting by the equivalences above them.)
    \item Finally, we introduce the type constructors, along with their introduction and elimination forms.
    The eliminators' inference rules are set up so they can act on substituted types; the introduction rules are as normal.

    Notably, we assume \emph{nothing} about how these interact with substitution.

    \todo{Is there a nicer (categorical?) explanation of these rules? It reminds me of the ``lax'' behavior found in higher category theory, but I wasn't able to figure it out.}
    \begin{itemize}
        \item \(\Pi\)-types:
        \begin{mathpar}
            \inferrule{A : \Ty(\Gamma) \\ B : \Ty(\Gamma, x : A)}{((x : A) \to B) : \Ty(\Gamma)}
            \and \inferrule{e : \Tm((\Gamma, x : A), B)}{\lambda x. e : \Tm(\Gamma, (x : A) \to B)}
            \and \inferrule{\sigma : \Sb(\Xi,\Gamma) \\ f : \Tm(\Xi, ((x : A) \to B)[\sigma])}{\lambda^{-1} x. f : \Tm((\Xi, x : A[\sigma]), B[\sigma[\mathsf{wk}],\mathsf{v}_0])}
            \\ \lambda^{-1} x. (\lambda x. e)[\sigma] \equiv e[\sigma[\mathsf{wk}], \mathsf{v}_0]
            \and f \equiv \lambda x. \lambda^{-1} x. f
        \end{mathpar}
        \item Universes:
        \begin{mathpar}
            \inferrule{ }{\mathcal{U} : \Ty(\Gamma)}
            \and \inferrule{A : \Ty(\Gamma)}{\mathsf{code}\,A : \Tm(\Gamma, \mathcal{U})}
            \and \inferrule{\sigma : \Sb(\Xi,\Gamma) \\ t : \Tm(\Xi, \mathcal{U}[\sigma])}{\mathsf{El}\,t : \Ty(\Xi)}
            \\ \mathsf{El}\,(\mathsf{code}\,A)[\sigma] \equiv A[\sigma]
            \and t \equiv \mathsf{code}\,\mathsf{El}\,t. t
        \end{mathpar}
    \end{itemize}
\end{itemize}

\subsection{Topos Interpretation}

To get an \emph{interesting} interpretation into toposes, we should be more careful about universe levels.

Interpret \(\Cx\) as the set of toposes. Interpret \(\Sb(\Gamma, \Delta)\)
as the set of geometric morphisms \(\Gamma \leftrightarrows \Delta\), acting as their inverse images.
Interpret \emph{small} types as objects of \(\Gamma\), and their terms as global elements.

The empty context, of course, is the topos of sets.
When constructing larger contexts as \(\Gamma, x : A\), require \(A\) to be small,
and interpret \(\Gamma, x : A\) as the slice topos \(\Gamma_{/A}\).

Then the constructions of substitutions check out:
\begin{itemize}
    \item There's always a unique geometric morphism into \(\mathbf{Set}\).
    \item Geometric morphisms \(\Xi \leftrightarrows \Gamma_{/A}\) correspond to pairs consisting of a geometric morphism \(\sigma : \Xi \leftrightarrows \Gamma\) and a global element \(\hom_\Xi(\mathbf{1},\sigma^*A)\).
    (\url{https://jenshemelaer.com/slice-toposes.pdf} seems to provide a proof, though I haven't checked it.)
\end{itemize}

You may observe that we can only reach contexts describing slice toposes.
This can be fixed by allowing large \(A\) in contexts \(\Gamma, x : A\).
I'm not sure how to do this properly, but at least \(\Gamma, x : \star\) ought to be the product in \(\mathbf{Topos}\)
of \(\Gamma\) and the classifying topos for the theory of objects.

Interpreting \(\Pi\)-types is easy: \((x : A) \to B\) should be the direct image of \(B\) under the geometric morphism \(\mathsf{wk} : \Sb((\Gamma, x : A),\Gamma)\).

\section{Consequences}

\subsection{Substitution And Eliminators}

By design, substitution does not commute through \(\Pi\)-types or \(\lambda\)-expressions.
However, it \emph{does} commute through \(\lambda^{-1}\), as a consequence of the \(\beta\) and \(\eta\) rules.
\[(\lambda^{-1} x. f)[\sigma[\mathsf{wk}],\mathsf{v}_0]
= \lambda^{-1} x. (\lambda x. (\lambda^{-1} x. f))[\sigma]
= \lambda^{-1} x. f[\sigma]
\]

When we derive function application from \(\lambda^{-1}\) in the usual way, we see that substitution commutes through it too.
\[f a = (\lambda^{-1} x. f)[\mathsf{id},a]\]
\begin{align*}
    (f a)[\tau] &= (\lambda^{-1} x. f)[\mathsf{id},a][\tau]
    \\ &= (\lambda^{-1} x. f)[\tau,a[\tau]]
    \\ &= (\lambda^{-1} x. f)[\tau[\mathsf{wk}],\mathsf{v}_0][\mathsf{id},a[\tau]]
    \\ &= (\lambda^{-1} x. f[\tau])[\mathsf{id},a[\tau]]
    \\ &= f[\tau] a[\tau]
\end{align*}

By similar logic, substitution commutes through \(\mathsf{El}\).

\subsection{Functoriality of Substitution}

The first thing to check is the functoriality of substitution.
If we can build a \(B\) from an \(A\), and we have an \(A[\sigma]\), we want a \(B[\sigma]\).

More formally, if we have \(\sigma : \Sb(\Gamma,\Delta)\),
and a type \(B : \Ty((\Delta, x : A))\) and a term \(e : \Tm((\Delta, x : A), B)\),
then we want to form a type \(B' : \Ty((\Gamma, x : A[\sigma]))\)
and a term \(e' : \Tm((\Gamma, x : A[\sigma]), B')\).

It took me embarassingly long to figure this out, but we're simply being asked for a substitution
\(\Sb((\Gamma, x : A[\sigma]), (\Delta, x : A))\).
And we have one. It's given by \((\sigma[\mathsf{wk}],\mathsf{v}_0)\).

\pagebreak

\listoftodos

\noindent And more:

\begin{itemize}
    \item Directed Univalence
    \item Do Church-encoded datatypes work? Assuming directed univalence, do we get the full induction principles, or just the recursion principles?
    \begin{itemize}
        \item If the latter, explore adding datatypes in the typical way. Substitutions should commute past many of them.
    \end{itemize}
    \item Just generally explore the system.
    \item Are there good higher-categorical models?
    \item Citations
\end{itemize}

\printbibliography

\end{document}